%%%%%%%%%%%%%%%%%%%%%%%%%%%%%%%%%%%%%%%%%%%%%%%%%%%%%%%%%%%%%%%%%%%
%%                                                               %%
%%                  This is an open project.                     %%
%%                     Written by Rocher.                        %%
%%                                                               %%
%%                                                               %%
%%%%%%%%%%%%%%%%%%%%%%%%%%%%%%%%%%%%%%%%%%%%%%%%%%%%%%%%%%%%%%%%%%%


%%%%%%%%%%%%%%%%%%%%%%%%%%%%%%%%%%%%%%%%%%%%%%%%%%%%%%%%%%%%%%%%%%%
%%                                                               %%
%%                           导言区                               %%
%%                                                               %%
%%%%%%%%%%%%%%%%%%%%%%%%%%%%%%%%%%%%%%%%%%%%%%%%%%%%%%%%%%%%%%%%%%%

%----------------指定编译模式为xeLaTeX,字符集为UTF-8------------------%
%!TEX program  = xelatex                                           
%!Mode:: "TeX:UTF-8"
\documentclass{ZUST_Thesis}
\usepackage{newtxtext}
\usepackage{newtxmath}

%%%%%%%%%%%%%%%%%%%%%%%%%%%%%%%%%%%%%%%%%%%%%%%%%%%%%%%%%%%%%%%%%%%%
%----------------                                ------------------%
%                                                                  %
%            !!!封面文件使用word生成pdf后加入文档!!!               %
%            !!!生成的pdf文件需要命名为titlepage!!!               %
%            !!!   封面文件存放入.tex同名目录   !!!               %
%                                                                  %
%%%%%%%%%%%%%%%%%%%%%%%%%%%%%%%%%%%%%%%%%%%%%%%%%%%%%%%%%%%%%%%%%%%%


\title{浙江科技学院硕士学位论文} 


%%%%%%%%%%%%%%%%%%%%%%%%%%%%%%%%%%%%%%%%%%%%%%%%%%%%%%%%%%%%%%%%%%%
%%                                                               %%
%%                         开始文档                               %%
%%                                                               %%
%%%%%%%%%%%%%%%%%%%%%%%%%%%%%%%%%%%%%%%%%%%%%%%%%%%%%%%%%%%%%%%%%%%
\begin{document}

	
%%%%%%%%%%%%%%%%%%%%%%%%%  生成标题页  %%%%%%%%%%%%%%%%%%%%%%%%%%%%%%%
\maketitle

%%%%%%%%%%%%%%%%%%%%%%%%%%  摘   要  %%%%%%%%%%%%%%%%%%%%%%%%%%%%%%%%%
%-----------------          中文摘要              -------------------%
\pagenumbering{Roman}
\pagestyle{abstractch}
\section*{摘\qquad 要}
\phantomsection\addcontentsline{toc}{section}{\bfseries 摘\qquad 要}

摘要的第一段主要阐述选题的意义。

第二段简单介绍研究动机、方法、目的。从存在的问题出发,阐述你采用什么方法,依据什么理论来做什么事,从而解决了前面指出的那些问题及取得的成果。本文的主要工作和成果如下:

1.针对....采用...,提出了...,效果如何;

2.500字左右;

摘要应表达学位论文的中心内容,简短明了,摘取论文中的基本信息,体现科研工作的核心思想。内容包括本文的目的和意义、研究方法、研究成果、结论。主要介绍自己的工作和取得的成果。根据内容自行分段描述。

最后一段,对全文进行总结,并对进一步的研究提出一些展望。

关键词数量为4-6个,每一关键词之间用逗号分开,最后一个关键词不用标点符号

\heiti{\textbf{关键词:}}\songti 关键词1,关键词2,关键词3,关键词4,关键词5

关键词数量为4-6个,每一关键词之间用逗号分开,最后一个关键词不用标点符号



%-----------------          英文摘要              -------------------%
\newpage
\pagestyle{abstracten}
\section*{ABSTRACT}
\phantomsection\addcontentsline{toc}{section}{\bfseries ABSTRACT}

The abstract in English goes here. Abstract in English should correspond to the  Chinese presented on the previous page. (英文摘要内容, 用小四号Times New Romans字体,每段开头留4个字符空格,英文摘要的内容应与中文摘要基本相对应)

\textbf{Key Words:} Key Words 1, Key Words 2, Key Words 3, Key Words 4

Key words in English go here and should be separated by a comma, but no comma is allowed after the last key word (关键词用小四号Times New Romans字体, 全部小写,每一关键词之间逗号分开,最后一个关键词后不打标点符号) 
%%%%%%%%%%%%%%%%%%%%%%%%%%%%%%%%%%%%%%%%%%%%%%%%%%%%%%%%%%%%%%%%%%%%





%%%%%%%%%%%%%%%%%%%%%%%%%%%  目   录  %%%%%%%%%%%%%%%%%%%%%%%%%%%%%%%
\newpage
\pagestyle{tabofcontent}
\tableofcontents
%%%%%%%%%%%%%%%%%%%%%%%%%%%%%%%%%%%%%%%%%%%%%%%%%%%%%%%%%%%%%%%%%%%%%



%%%%%%%%%%%%%%%%%%%%%%%%%%%%%%%%%%%%%%%%%%%%%%%%%%%%%%%%%%%%%%%%%%%%%
%%%%%%%%%%%%%%%%%%%%%%%%%%%%%%%%%%%%%%%%%%%%%%%%%%%%%%%%%%%%%%%%%%%%%
%%%%%%%%%%%%%%%%%%%%%%%%%%%  正   文  %%%%%%%%%%%%%%%%%%%%%%%%%%%%%%%%
%%%%%%%%%%%%%%%%%%%%%%%%%%%%%%%%%%%%%%%%%%%%%%%%%%%%%%%%%%%%%%%%%%%%%
%%%%%%%%%%%%%%%%%%%%%%%%%%%%%%%%%%%%%%%%%%%%%%%%%%%%%%%%%%%%%%%%%%%%%
\newpage
\pagestyle{main}
\pagenumbering{arabic}
\setcounter{page}{1}
%%%%%%%%%%%%%%%%%%%%%%%%%%%%%%%%%%%%%%%%%%%%%%%%%%%%%%%%%%%%%%%%%%%%%
%%%%%%%%%%%%%%%%%%%%%%%%%%%%%%%%%%%%%%%%%%%%%%%%%%%%%%%%%%%%%%%%%%%%%
%%%%%%%%%%%%%%%%%%%%%%%%%%% 自定义符号 %%%%%%%%%%%%%%%%%%%%%%%%%%%%%%%%
%%%%%%%%%%%%%%%%%%%%%%%%%%%%%%%%%%%%%%%%%%%%%%%%%%%%%%%%%%%%%%%%%%%%%
%%%%%%%%%%%%%%%%%%%%%%%%%%%%%%%%%%%%%%%%%%%%%%%%%%%%%%%%%%%%%%%%%%%%%

\setlength\abovedisplayskip{0em}
\setlength\belowdisplayskip{0em}

\section{引言}


\subsection{国内外研究现状}


这是引用的参考文献\cite{CNN},参考文献\cite{DNN,深度学习,西瓜书},以及参考文献\ncite{resnet}。



\subsection{本文研究内容与组织结构}








\section{方法}

\begin{figure}[h]
	\centering
	\includegraphics[width=0.4\linewidth]{figure/mathtype2.jpg}   
	\caption{示意图}
	\label{fig_example}
\end{figure}


\begin{figure}[h]
	\centering
	\subfigure[子图1标题]{
		\includegraphics[width=0.3\linewidth]{figure/mathtype2.jpg}}
	\subfigure[子图2标题]{
		\includegraphics[width=0.3\linewidth]{figure/mathtype2.jpg}}
	\caption{多图示意}
	\label{fig_subexample}
\end{figure}

\begin{table}[h]
	\small
	\setlength{\abovecaptionskip}{0em}
	\setlength{\belowcaptionskip}{0.5em}
	\renewcommand\arraystretch{1.5}
	\setlength\tabcolsep{3pt}
	\centering
	\caption{示例表}
	\label{tab_example}
	\begin{tabular}{m{4cm}<{\centering}m{3cm}<{\centering}m{2cm}<{\centering}}
		\toprule
		表格内内容选单倍行距,在“格式-段落”的段前段后选3磅&总的原则是美观&第一行\\
		\midrule
		在“表格-表格属性”中选居中&$\pm\Omega_i$&$\mp\Omega_i$\\
		第一列&$\pm\Omega_i$&$\mp\Omega_i$\\
		\bottomrule
	\end{tabular}
\end{table}





%%%%%%%%%%%%%%%%%%%%%%%%%%%  附   录  %%%%%%%%%%%%%%%%%%%%%%%%%%%%%%%
\newpage
\pagestyle{appendix}
\phantomsection\addcontentsline{toc}{section}{\bfseries 参考文献}
\renewcommand{\refname}{参\quad 考\quad 文\quad 献}
\setlength{\bibsep}{0em}
\bibliographystyle{gbt7714-numerical}
\begin{spacing}{1.25} 
	\bibliography{ref.bib}
\end{spacing}
%%%%%%%%%%%%%%%%%%%%%%%%%%%%%%%%%%%%%%%%%%%%%%%%%%%%%%%%%%%%%%%%%%%%%



%%%%%%%%%%%%%%%%%%%%%%%%%%%  致    谢  %%%%%%%%%%%%%%%%%%%%%%%%%%%%%%%
\newpage
\pagestyle{acknowledgement}
\section*{致\qquad 谢}
\phantomsection\addcontentsline{toc}{section}{\bfseries 致\qquad 谢}
对给予各类指导、支持、和协助完成研究工作,以及提供各种条件的单位及个人表示感谢。致谢应实事求是,切忌浮夸与庸俗之词。 
%%%%%%%%%%%%%%%%%%%%%%%%%%%%%%%%%%%%%%%%%%%%%%%%%%%%%%%%%%%%%%%%%%%%%





%%%%%%%%%%%%%%%%%%%%%%%%%%%   科研成果   %%%%%%%%%%%%%%%%%%%%%%%%%%%%%%%
\newpage
\pagestyle{achievement}
\section*{攻读学位期间参加的科研项目和成果}
\phantomsection\addcontentsline{toc}{section}{\bfseries 攻读学位期间参加的科研项目和成果}

本人在攻读研究生学位期间获得的科研成果有:


%%%%%%%%%%%%%%%%%%%%%%%%%%%%%%%%%%%%%%%%%%%%%%%%%%%%%%%%%%%%%%%%%%%%%


%%%%%%%%%%%%%%%%%%%%%%%%%%%%%%%%%%%%%%%%%%%%%%%%%%%%%%%%%%%%%%%%%%%%%


\end{document}